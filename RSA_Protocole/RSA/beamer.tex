\documentclass[10pt]{beamer}
%%%%%%%%%% Beamer 
%\usepackage{beamerthemesplit}
%\useoutertheme{infolines}
% \useinnertheme{default}
%\useoutertheme{shadow}
% \usepackage{float,tabls}
% \usepackage[T1]{fontenc}
\usepackage[latin1]{inputenc}
 \usepackage[francais]{babel}
% \usepackage{color}
\usepackage{amsfonts,amssymb,amsthm,amsmath}
\usepackage{graphics,graphicx}
\usepackage{aeguill}
%\usepackage{bbm}
\usepackage{url}
\usefonttheme[onlymath]{serif}
\usepackage{dsfont}
\usepackage{ulem}
\usepackage{stmaryrd}
%\usepackage{mathabx}
% \usepackage{extsizes}
\usepackage[Algorithme]{algorithm}
\usepackage{algorithmic}
\usepackage{listing}

\theoremstyle{plain}
\newtheorem{theoreme}{{Théorème}}[section] 
\newtheorem*{theoreme*}{{Théorème}}
\newtheorem{proposition}[theoreme]{Proposition}
\newtheorem*{proposition*}{Proposition}
\newtheorem{corollaire}[theoreme]{Corollaire}
\newtheorem*{corollaire*}{Corollaire}
\newtheorem{lemme}[theoreme]{Lemme}
\newtheorem*{lemme*}{Lemme}

\theoremstyle{definition}
%\newtheorem{definition}[theoreme]{Définition}
\newtheorem*{definition*}{Définition}

\theoremstyle{remark}
\newtheorem*{remarque*}{Remarque}
\newtheorem{remarque}[theoreme]{Remarque}
\newtheorem{exemple}[theoreme]{Exemple}
\newtheorem*{exemple*}{Exemple}
\newtheorem{exemples}[theoreme]{Exemples}
\newtheorem*{exemples*}{Exemples}

\newcommand{\comm}[1] {{\bf (#1)}}
\newcommand{\commm}[1]{{}}


% Alphabet grec
\renewcommand{\a}{\alpha}\renewcommand{\b}{\beta}\newcommand{\g}{\gamma}\newcommand{\G}{\Gamma}\renewcommand{\d}{\delta}\newcommand{\D}{\Delta}\newcommand{\e}{\varepsilon}\newcommand{\eps}{\epsilon}\newcommand{\z}{\zeta} \newcommand{\y}{\eta}\renewcommand{\th}{\theta}\newcommand{\Th}{\Theta}\renewcommand{\i}{\iota}\renewcommand{\k}{\kappa}\newcommand{\vk}{\varkappa}\renewcommand{\l}{\lambda}\renewcommand{\L}{\Lambda}\newcommand{\m}{\mu}\newcommand{\n}{\nu}\newcommand{\x}{\xi}\newcommand{\X}{\Xi}\newcommand{\s}{\sigma}\renewcommand{\SS}{\Sigma}\renewcommand{\t}{\tau}\newcommand{\f}{\varphi}\newcommand{\vf}{\phi}\newcommand{\h}{\chi}\newcommand{\p}{\psi}\renewcommand{\P}{\Psi}\renewcommand{\o}{\omega}\renewcommand{\O}{\Omega}

\newcommand{\R}{\mathbb{R}}	\newcommand{\Q}{\mathbb{Q}}	\newcommand{\C}{\mathbb{C}}
\newcommand{\N}{\mathbb{N}}	\newcommand{\Z}{\mathbb{Z}}


\setbeamertemplate{navigation symbols}{% Permet de choisir les icones de navigation en bas du transparent
% \insertslidenavigationsymbol
% \insertframenavigationsymbol
% \insertsubsectionnavigationsymbol
% \insertsectionnavigationsymbol
% \insertdocnavigationsymbol
% \insertbackfindforwardnavigationsymbol
}

% TITRE DU DOCUMENT
\title[RSAES-OAEP et RSASSA-PSS]{Présentation des protocoles\\ RSAES-OAEP et RSASSA-PSS}
\subtitle{M2 MIC - Cryptographie asymétrique}
\author[J. Nekam et D. Resende]{Jérémie Nekam et Daniel Resende}
\institute[Paris Diderot]{\includegraphics[scale=0.5]{upd.jpg}}
\date{Mardi 24 octobre 2017}

\usetheme{Madrid}
\justifying

\begin{document}

\frame{\titlepage}

\begin{frame}
\transwipe
\frametitle{Sommaire}
\tableofcontents[pausesections]
\end{frame}

\AtBeginSection[]
{
	\begin{frame}
	\transwipe
	\frametitle{Sommaire}
	\tableofcontents[currentsection, hideothersubsections]
	\end{frame}
}


%%%%%%%%%%%%%%%%%%%%%%%%%%%%%%%%%%%%%%%%%%%%%%%%%%%%%%%%%%%%%%%%%%%%%
\section{Introduction}
\begin{frame}
\frametitle{Introduction}
\transwipe
Deux protocoles pour deux utilisations différentes :
\begin{description}
\pause
\item[RSAES-OAEP] Protocole de chiffrement/déchiffrement
\pause
\item[RSASSA-PSS] Protocole de signature
\end{description}
\end{frame}
%%%%%%%%%%%%%%%%%%%%%%%%%%%%%%%%%%%%%%%%%%%%%%%%%%%%%%%%%%%%%%%%%%%%%
\section{RSAES-OAEP}
\begin{frame}
\transwipe
Le protocole RSAES-OAEP se décompose en deux parties :
\begin{itemize}
\item EM-OAEP
\item RSAEP (resp. RSADP) pour le chiffrement (resp. déchiffrement)
\end{itemize}
\end{frame}
%%%%%%%%%%%%%%%%%%%%%%%%%%%%%%%%%%%%%%%%%%%%%%%%%%%%%%%%%%%%%%%%%%%%%
\begin{frame}
\transwipe
\frametitle{Pourquoi utiliser OAEP ?}
D. Bleichembacher a trouvé une attaque CCA-2 sur le protocole suivant :

\begin{definition}[PKCS 1 v1]
Soit $M$ le message à chiffrer.
On note \struct{$EB = 00\parallel 02\parallel Padding\parallel 00\parallel M$}
\end{definition}
\end{frame}
%%%%%%%%%%%%%%%%%%%%%%%%%%%%%%%%%%%%%%%%%%%%%%%%%%%%%%%%%%%%%%%%%%%%%
\subsection{OAEP}
\begin{frame}
\transwipe
\frametitle{Le schéma OAEP standard}
\begin{algorithm}[H]
\caption{Schéma OAEP}
\label{oaep}
\begin{algorithmic} 
\REQUIRE Un message $m$, un aléa $r$ et deux oracles $G$ et $H$.
\ENSURE Un message $m'$ tel que $m' = s\parallel t$.
\end{algorithmic}
\end{algorithm}
\begin{figure}[H]
\centering
\includegraphics[scale=0.8]{OAEP.png}
\caption{OAEP}
\end{figure}
\end{frame}
%%%%%%%%%%%%%%%%%%%%%%%%%%%%%%%%%%%%%%%%%%%%%%%%%%%%%%%%%%%%%%%%%%%%%
\subsection{Génération des clés RAES-OAEP}
\begin{frame}
\frametitle{Génération des clés RAES-OAEP}
\transwipe
\begin{block}{Clés publiques}
On garde les mêmes clés $(n, e)$ avec les mêmes propriétés que le RSA classique.
\end{block}
\begin{block}{Clés privées}	
\begin{itemize}
	\item soit \structure{$(p, q, d)$} tel que $e\cdot d = 1\ mod\ (ppcm(p - 1, q - 1))$,
	\item soit \structure{$(p, q, dP, dQ, qInv)$} où $q\cdot qInv = 1\ mod\ p$, $e\cdot dP = 1\ mod\ q$ et $e\cdot dP = 1\ mod\ q$.
	\end{itemize}
\end{block}
\end{frame}
%%%%%%%%%%%%%%%%%%%%%%%%%%%%%%%%%%%%%%%%%%%%%%%%%%%%%%%%%%%%%%%%%%%%%
\subsection{Utilisation d'OAEP avec RSA}
\begin{frame}
\transwipe
\begin{algorithm}[H]
\caption{Schéma EM-OAEP}
\label{oaep}
\begin{algorithmic} 
\REQUIRE Un message $m$, un aléa $seed$ et $Hash$ des données spécifiant la fonction de hachage à utiliser
\ENSURE Un message $EM$.
\end{algorithmic}
\end{algorithm}
\frametitle{Le schéma EM-OAEP}
\begin{figure}[H]
\centering
\includegraphics[scale=0.35]{EM-OAEP.png}
\caption{EM-OAEP}
\end{figure}
\end{frame}
%%%%%%%%%%%%%%%%%%%%%%%%%%%%%%%%%%%%%%%%%%%%%%%%%%%%%%%%%%%%%%%%%%%%%
\subsection{Chiffrement/déchiffrement de RAES-OAEP}
\begin{frame}
\frametitle{Chiffrement/déchiffrement de RAES-OAEP}
\transwipe
\begin{block}{RSAEP - Chiffrement}
On garde les mêmes paramètres et propriétés que le RSA classique.
\end{block}
\begin{algorithm}[H]
\caption{RSADP - Déchiffrement}
\label{rsadp}
\begin{algorithmic} 
\REQUIRE Un message chiffré $c$ et une clé privé $K = (n, p, q, d)$ ou $(p, q, dP, dQ, qInv)$. 
\ENSURE Un message clair $m$
\IF{$c$ n'est pas une entrée valide} 
    \RETURN ERREUR
\ENDIF
\IF{$K = (n, p, q, d)$}
	\RETURN $m = c^{d}\ mod\ n$
\ENDIF
\STATE $m_{1} = c^{dP}\ mod\ p$
\STATE $m_{2} = c^{dQ}\ mod\ q$
\STATE $h = (m_{1} - m_{2})\cdot qInv\ mod p$
\RETURN $m = m_{2} + q\cdot h$

\end{algorithmic}
\end{algorithm}
\end{frame}
%%%%%%%%%%%%%%%%%%%%%%%%%%%%%%%%%%%%%%%%%%%%%%%%%%%%%%%%%%%%%%%%%%%%%
\subsection{Sécurité du protocole}
\begin{frame}
\frametitle{Sécurité du protocole}
\transwipe
\begin{definition}[Sécurité sémantique]
Soit $m_{0}, m_{1}$ deux messages choisies par l'attaquant.
Soit $c$ un challenge qui est le chiffré de $m_{0}$ ou $m_{1}$. On dit qu'un protocole est sémantiquement sûr si l'attaquant ne peut pas distinguer $m_{0}$ ou $m_{1}$.
\end{definition}
\end{frame}
%%%%%%%%%%%%%%%%%%%%%%%%%%%%%%%%%%%%%%%%%%%%%%%%%%%%%%%%%%%%%%%%%%%%%
\begin{frame}
\frametitle{Attaque de Shoup}
\transwipe
\begin{proposition}
	Le protocole f-OAEP n'est pas totalement sémantiquement sûr. 
\end{proposition}
\end{frame}
%%%%%%%%%%%%%%%%%%%%%%%%%%%%%%%%%%%%%%%%%%%%%%%%%%%%%%%%%%%%%%%%%%%%%
\begin{frame}[shrink]
\frametitle{Idées de preuve pour l'attaque de Shoup}
\transwipe
\begin{definition}[Xor-malléable]
Soit $f$ une permutation à sens unique avec trappe.
On dit que $f$ est \structure{xor-malléable}, si on a une probabilité non-négligeable de pouvoir calculer $f(t\oplus a)$ en connaissant $f(t)$ et $a$.
\end{definition}
Soient $c$ un chalenge, $H$ un oracle aléatoire et $f$ telle que $f(s\parallel t) = s\parallel f(t)$ une fonction xor-maléable.
\begin{enumerate}
\item On pose $c = f(s\parallel t)= s\parallel f(t)$.
\item On choisi aléatoirement $s'$ et on calcul $a=H(s)\oplus H(s')$
\end{enumerate}
\end{frame}
%%%%%%%%%%%%%%%%%%%%%%%%%%%%%%%%%%%%%%%%%%%%%%%%%%%%%%%%%%%%%%%%%%%%%
\section{RSASSA-PSS}
\subsection{PSS}
\begin{frame}
\transwipe

\end{frame}
%%%%%%%%%%%%%%%%%%%%%%%%%%%%%%%%%%%%%%%%%%%%%%%%%%%%%%%%%%%%%%%%%%%%%
\subsection{Utilisation de PSS avec RSA}
\begin{frame}
\transwipe

\end{frame}
%%%%%%%%%%%%%%%%%%%%%%%%%%%%%%%%%%%%%%%%%%%%%%%%%%%%%%%%%%%%%%%%%%%%%
\subsection{Sécurité du protocole}
\begin{frame}
\transwipe

\end{frame}
%%%%%%%%%%%%%%%%%%%%%%%%%%%%%%%%%%%%%%%%%%%%%%%%%%%%%%%%%%%%%%%%%%%%%
\section{Conclusion/Recommandation}
\begin{frame}
\transwipe
\begin{description}
\item[OAEP] Il est préférable de plus utiliser OAEP, et plutôt REACT.
\item[PSS]
\end{description}

\end{frame}
%%%%%%%%%%%%%%%%%%%%%%%%%%%%%%%%%%%%%%%%%%%%%%%%%%%%%%%%%%%%%%%%%%%%%
\begin{frame}[shrink]
\frametitle{Bibliographie}
\nocite{*}
\bibliographystyle{alpha}
\bibliography{Bibliography}
\end{frame}


\end{document}
