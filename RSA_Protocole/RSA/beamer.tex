\documentclass[12pt]{beamer}
\usetheme{Madrid}
\usepackage[french]{babel}
\usepackage[utf8]{inputenc}
\usepackage{textcomp}
\usepackage[T1]{fontenc}
\usepackage{epsfig}
\usepackage{wrapfig} %for figures
\usepackage{xcolor} %for color
\usepackage{ragged2e}
\usepackage{amsfonts}
\usepackage{amssymb}
\usepackage{amsmath}
\usepackage[Algorithme]{algorithm}
\usepackage{algorithmic}
\usepackage{listing}
\usepackage{hyperref}
\hypersetup{pdfstartview=XYZ}
\justifying

% TITRE DU DOCUMENT
\title[RSAES-OAEP et RSASSA-PSS]{Présentation des protocoles\\ RSAES-OAEP et RSASSA-PSS}
\subtitle{M2 MIC - Cryptographie asymétrique}
\author[J. Nekam et D. Resende]{Jérémie Nekam et Daniel Resende}
\institute[Paris Diderot]{\includegraphics[scale=0.5]{upd.jpg}}

\date{Mardi 24 octobre 2017}


%%%%%%%%%%%%%%%%%%%%%%%%%%%%%%%%%%%%%%%%%%%%%%%%%%%%%%%

\begin{document}

\begin{frame}
\titlepage
\end{frame}

\begin{frame}
\transwipe
\frametitle{Sommaire}
\tableofcontents[pausesections]
\end{frame}

\AtBeginSection[]
{
	\begin{frame}
	\transwipe
	\frametitle{Sommaire}
	\tableofcontents[currentsection, hideothersubsections]
	\end{frame}
}


%%%%%%%%%%%%%%%%%%%%%%%%%%%%%%%%%%%%%%%%%%%%%%%%%%%%%%%%%%%%%%%%%%%%%
\section{Introduction}
\begin{frame}
\transwipe
Deux protocoles pour deux utilisations différentes :
\begin{description}
\pause
\item[RSAES-OAEP] Protocole de chiffrement
\pause
\item[RSASSA-PSS] Protocole de signature
\end{description}
\end{frame}
%%%%%%%%%%%%%%%%%%%%%%%%%%%%%%%%%%%%%%%%%%%%%%%%%%%%%%%%%%%%%%%%%%%%%
\section{RSAES-OEAP}
\begin{frame}
\transwipe
Le protocole RSAES-OAEP se décompose en deux parties :
\begin{itemize}
\item EM-OAEP
\item RSAEP (resp. RSADP) pour le chiffrement (resp. déchiffrement)
\end{itemize}
\end{frame}
%%%%%%%%%%%%%%%%%%%%%%%%%%%%%%%%%%%%%%%%%%%%%%%%%%%%%%%%%%%%%%%%%%%%%
\subsection{OAEP}
\begin{frame}
\transwipe
\frametitle{Le schéma OAEP standard}
\begin{figure}[H]
\centering
\includegraphics[scale=1]{OAEP.png}
\caption{OAEP}
\end{figure}
\end{frame}
%%%%%%%%%%%%%%%%%%%%%%%%%%%%%%%%%%%%%%%%%%%%%%%%%%%%%%%%%%%%%%%%%%%%%
\subsection{EM-OAEP : Utilisation d'OAEP avec RSA}
\begin{frame}
\transwipe
Entrées du schéma:
\begin{description}
\item[Hash] Données spécifiant la fonction de hachage
\item[M] Message
\item[PS] Padding
\item[Seed] Aléa
\end{description}
\frametitle{Le schéma EM-OAEP}
\begin{figure}[H]
\centering
\includegraphics[scale=0.38]{EM-OAEP.png}
\caption{OAEP}
\end{figure}
\end{frame}
%%%%%%%%%%%%%%%%%%%%%%%%%%%%%%%%%%%%%%%%%%%%%%%%%%%%%%%%%%%%%%%%%%%%%
\subsection{Sécurité du protocole}
\begin{frame}
\transwipe

\end{frame}
%%%%%%%%%%%%%%%%%%%%%%%%%%%%%%%%%%%%%%%%%%%%%%%%%%%%%%%%%%%%%%%%%%%%%
\section{RSASSA-PSS}
\subsection{PSS}
\begin{frame}
\transwipe

\end{frame}
%%%%%%%%%%%%%%%%%%%%%%%%%%%%%%%%%%%%%%%%%%%%%%%%%%%%%%%%%%%%%%%%%%%%%
\subsection{Utilisation de PSS avec RSA}
\begin{frame}
\transwipe

\end{frame}
%%%%%%%%%%%%%%%%%%%%%%%%%%%%%%%%%%%%%%%%%%%%%%%%%%%%%%%%%%%%%%%%%%%%%
\subsection{Sécurité du protocole}
\begin{frame}
\transwipe

\end{frame}
%%%%%%%%%%%%%%%%%%%%%%%%%%%%%%%%%%%%%%%%%%%%%%%%%%%%%%%%%%%%%%%%%%%%%
\section{Conclusion}
\begin{frame}
\transwipe

\end{frame}
%%%%%%%%%%%%%%%%%%%%%%%%%%%%%%%%%%%%%%%%%%%%%%%%%%%%%%%%%%%%%%%%%%%%%
\begin{frame}[shrink]
\frametitle{Bibliographie}
\nocite{*}
\bibliographystyle{alpha}
\bibliography{Bibliography}
\end{frame}
\end{document}